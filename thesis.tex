\documentclass{article}

\usepackage{caption}
\usepackage{code}
\usepackage{geometry}
\usepackage{graphicx}
\usepackage{multirow}
\usepackage{url}
\usepackage{verbatim}
\usepackage{fancyvrb}
\usepackage{booktabs}
\usepackage{tabularx}
\usepackage{color}
\usepackage{colortbl}

\definecolor{Gray}{gray}{0.9}
\newcommand\ch[1]{\textcolor[rgb]{0,0,1}{\textbf{#1}}}
\newcommand\clh[1]{\textcolor[rgb]{0,.5,1}{\textbf{#1}}}
\newcommand\coh[1]{\textcolor[rgb]{0,.6,0}{\textbf{#1}}}

\newtheorem{research-question}{Research Question}[section]

\newcommand{\todo}[1]{{\color{red}\bfseries [[#1]]}}

\title{Celeriac: A Daikon .NET Front End}
\author{Kellen Donohue\\ \\
University of Washington, Computer Science \& Engineering\\ \\}
\begin{document}
\maketitle

\section{Introduction}
\subsection{Contracts}
Contracts provide an agreement between a method implementer and caller, and also describe well-formed object. The following code snippet shows how a typical bank account withdraw method may be annotated with contracts.
\begin{center}
\begin{Verbatim}[commandchars=\\\{\}]

\ch{class} BankAccount \{
  \ch{bool} Withdraw(\clh{Account} acct, \ch{decimal} amt) \{
      \coh{// Don't withdraw a negative amount.}
      \clh{Contract}.Requires(amt >= 0);
      \coh{// Return whether withdraw succeeded.}
      \clh{Contract}.Ensures(
        \clh{Contract}.Result<\ch{bool}>() == acct.bal >= amt);
      \coh{// Update the account properly.}
      \clh{Contract}.Ensures(acct.bal ==
        \clh{Contract}.OldValue(acct.bal) - \clh{Contract}.Result<\ch{bool}>()
        ? amt : 0);
      ...
  \}
\}
\end{Verbatim}
\end{center} \captionof{figure}{Bank Account With Contracts}
\  \\
Contracts are useful for ensuring the correct behavior of programs. Additionally, the .NET framework has many tools that are enhanced by the use of contracts, in a framework called CodeContracts. The Pex white-box test generation tool \cite{pex} won't generate tests that violate specified contracts, which is important for ensuring that generated tests satisfy a specification. Developers can use contracts to statically verify program properties \cite{static} and automatically generate API documentation \cite{contracts}.

\subsection{Daikon}
To help developers insert appropriate contracts the Daikon tool \cite{daikon} dynamically detects likely program invariants. It takes as input a program trace, and outputs a list of possible invariants based on that trace.  The following figure demonstrates a code snippet and the possible invariants that Daikon may infer on the snippet.

\begin{center}
\begin{Verbatim}[commandchars=\\\{\}]
\ch{public void} makeEmpty() \{
    \clh{Arrays}.fill(theArray, 0, topOfStack + 1, \ch{null});
    topOfStack = -1;
\}

-----------------------------------------------------

Daikon Invariants:
DataStructures.StackAr.makeEmpty():::EXIT
size(this.theArray[]) == orig(size(this.theArray[]))
this.theArray[] elements == null
this.topOfStack == -1
this.theArray[0..this.topOfStack] == []
this.topOfStack <= orig(this.topOfStack)

\end{Verbatim}
\end{center} \captionof{figure}{Daikon Example}
\  \\
In this example all the Daikon inferred invariants are true of the source code, however that is not always the case. Since Daikon can only output invariants based on input traces it may output invariants that are too weak (since it is limited by the type of invariants that may occur) or too strong (since it may over-generalize if given a trace that doesn't well demonstrate the program).
\\ \\
Daikon works as follows, and as described in the figure below. First, front-end modifies a program to add instrumentation calls which output a data trace during execution. During this process a declaration file, or description of the program schema in terms of methods and variables is output. The datatrace created during execution along with the declaration file are then run through Daikon, which outputs a list of possible program invariants. These invariants can be inserted as contracts, assertions, or documentation in the source program, among a variety of other possible uses.

\begin{center}
\includegraphics[scale=.5]{daikon}
\end{center} \captionof{figure}{Operation of Daikon}

\subsection{Contributions}
The contribution of this work is as follows. The technical contribution is Celeriac, a Daikon front end for the .NET programming languages C\#, F\#, and VB.NET. The research contribution is an investigation of which types of contracts developers choose to include in their programs, and why. Section 2 described how Celeriac relates to other Daikon front ends, and other .NET profiler tools. Section 3 describes the use of Celeriac. Section 4 describes the implementation of Celeriac, including important design contributions. Section 5 describes the application of Celeriac to answer software engineering questions. Finally, Section 6 concludes with a summary of the contributions and future work.

\section{Related Work}
Daikon .NET front ends exist for some other languages. The Chicory front end, for Java, is very popular, and Celeriac was modeled after the external behavior or Chicory in many cases. The Kvasir front end, for C++, was a model for the testing suite as well.
\\ \\
Celeriac is unique among Daikon front ends in that no other front end has been written for a functional language, such as F\#. Also, no other front ends have explicitly targeted a multi-language platform, such as .NET.

\section{User Documentation}

\section{Developer Documentation}

\section{Research Experiments}
The primary research question Celeriac answers is what kinds of contracts are inferred from real programs and libraries. Important related research questions are, which of these inferred contracts do developers choose to include in their subject. To answer these questions a study was done on programs C\# open source programs that contained Code Contracts.

\subsection{Subject Programs}
\label{sec:subject-programs}

This subsection describes the four subject programs and their
current use of the Code Contract tool as described by a developer
questionnaire. We selected these subject projects from the ``Projects
Using Code Contracts'' listing on the Microsoft Code Contracts
project
website\footnote{https://research.microsoft.com/en-us/projects/contracts/}.
%
The projects were chosen because they are open-source, actively
developed, have users (i.e., are not toy programs), and use contracts
in a meaningful way.

\subsubsection{Labs Framework}

The Labs Framework\footnote{https://labs.codeplex.com/} (11K SLOC) is
a framework for managing ``experiments'' exploring the behavior of an
API are library. Labs provide a means for demonstrating features to a
library's users, or for a user to perform reproducible
experimentation.
\\ \\
The project uses \verb|ccheck|, the static Code Contract checker,
which is enabled by default for both the Debug and Release
configurations (but not the Profiling configuration).

\subsubsection{Mishra Reader}

Mishra Reader\footnote{\url{http://mishrareader.codeplex.com/}}
(19K SLOC) is an open-source Google Reader client with over 27,500
downloads.
%
The developer we interviewed introduced Code Contracts to the core
library 2 1/2 years ago (Fall 2011) to help reduce bugs in
multi-threaded code.
% Before this study, the codebase included 59 preconditions, 14 postconditions, and 4
% object invariants. 69 of these contracts were nullness checks on
% either an argument or method return value; the project did not include quantification
% (\verb|ForAll| or \verb|Exists|) contracts.
%
Code Contracts were written after the methods were written to aide in
debugging (i.e., as opposed to design by contract).
%
Initially satisfied with Code Contracts, the developer became
dissatisfied due to a lack of support for debugging with contract in
\verb|async| and \verb|await| constructs (the IL rewriter did not
properly modify the debugging information); debugging support for
these constructs has since been added.

The development team does not use the static Code Contract checker,
citing that it is slow and noisy. Runtime checks are configured to
throw an exception and terminate the program; the project's test suite
is also set up to fail a test given a contract violation.

\begin{table}[t]
\begin{center}{
\begin{tabular}{ ll }
  Project & Code Contract Use \\ \hline
  Labs Framework & Static checking \\
  Mishra Reader & Debugging concurrent code \\
  Sando & Early runtime error detection \\
  QuickGraph & Static checking and Pex
\end{tabular} }
\end{center}
\caption{The subject programs and their primary use of Code Contracts}
\end{table}

\subsubsection{Sando}

Sando\footnote{https://sando.codeplex.com/} is a Lucene-based code
search engine that includes a Visual Studio interface; the software is
Beta with over 500 downloads.
%
The Sando project has used Code Contracts for 1 1/2 years (since
Winter 2012). Code Contracts were introduced to the project since one
of main contributors had seen a webinar on Code Contracts and wanted
to try them out. The team found the initial project setup more
difficult than expected, especially due to issue integrating Code
Contracts into the automated TeamCity build. Despite these initial
challenges, the developer we interviewed feels that Code Contracts has
sped the discovery of bugs or regressions, as well as increasing
confidence in the quality of code containing contracts. Code Contracts
are enabled only for debug builds.
% The Sando project does not include any XML documentation, instead
% using method and parameter naming as the primary documentation (along
% with some comments within methods).

The team primarily uses Code Contracts in core functionality,
primarily in the Index component, as placing bad data into the index
can result in later errors. The interview developer typically writes
contracts after making a change, but before running the unit test
suite before check-in.

The interviewed developer was not aware of the static analysis tools
Code Contracts. The project does not use any static analysis tools, in
part because the team has limited build engineering resources. Code
Contracts is seen as offering additional quality assurance without
requiring additional build engineering, and likely makes the team less
likely to try other quality assurance tools.

\subsubsection{Quick Graph}

Quick Graph\footnote{https://quickgraph.codeplex.com/} (32K SLOC) is
a data structure and algorithm library. The project uses both the Code
Contract checker, and PEX.


\subsection{Observable Contracts}
\label{sec:celeriac}

In this section, we report on an analysis of the observable behaviors
of the subject programs.
%
In the absence of written documentation and contracts, developers
often infer the ``contract'' for a program by reasonably generalizing
the program's behavior.

To guide our analysis we posed the following three questions which
build on the results for written specification in the previous section:

\begin{research-question}
  To what extent are observable behaviors domain- or
  application-independent?
\end{research-question}

From previous work on pluggable types~\cite{DietlDEMS2011} (and
corroborated in Section~\ref{sec:survey}), we expect the distribution
of behaviors to be tied to the application-domain.

\begin{research-question}
  Are Code Contracts able to concisely express commonly observable
  behaviors?
\end{research-question}

Since Code Contracts are written as method calls at the beginning of
a method, they can be verbose for short properties (e.g.,
nullness). Additionally, the limits of C\#'s type inference mean that
expressing contracts involving the result of a method are verbose
since the return expression must be parameterized by the method's
return type.
%
The Code Contracts team recently introduced ``Argument Validator'' and
``Contract Abbreviator'' methods to make expressing common patterns
more concise; we wanted validate their impact on expressing common
program behaviors.

\begin{research-question}
  What are differences (qualitative and quantitative) between the Code
  Contracts that are written in the subject program, and the observable
  behavior of the program?
\end{research-question}

From the data in Section~\ref{sec:survey} (and out own experience), we
hypothesized that developers disproportionately write generic, or
simple, generic contracts. Additionally, Polikarpova et al. note that
developers are typically worse at writing postconditions than
preconditions~\cite{Polikarpova2012}.

\subsubsection{Methodology}

We exercised the behavior of Quick Graph using its unit tests; Sando
using its integration test suite; Mishra Reader by using the
application; and Labs by running the Labs for the Rxx
project~\footnote{https://rxx.codeplex.com/}.

As a proxy for the behaviors that a developer could observe by calling
an API, we used the Daikon invariant detector~\cite{daikon}
to infer invariants. Daikon takes as input one or more execution
traces, and reports data properties that were true over the observed
executions.

Daikon employs statistical methods (e.g., minimum support and
confidence heursistics) to infer likely method preconditions, method
postconditions, and object invariants.
%
The contracts that Daikon infers are sound with respect to the
observed executions --- i.e., it does not infer any properties that
are falsified by one of the observed traces.
%
We address the shortcomings of Daikon as they relate to this study in
Section~\ref{sec:celeriac-threats}.

\subsubsection{Abstract Type Inference}
\label{sec:comparability}

By default, Daikon attempts to compare all values that have the same
internal representation in Daikon. Since reference types are all
represented in Daikon using their hashcode, Daikon will report
comparisons between every reference value (including those in
violation of the language's typing rules).
%
To reduce the number of spurious contracts reported, Daikon supports
comparability sets for the expressions at each program point (object,
method entry, and method exit).
%
A comparability set defines a group of related variables. For example,
in a program with \verb|int|s, some variables may represent (have an
abstract type of) months, days, or years. Given comparability
information, Daikon would only output invariants relating expressions
with the same abstract type.

Existing Daikon comparability analyses for Java and C/C++ programs are
dynamic, recording variable interactions at runtime
(e.g., DynComp~\todo{cite}). For Celeriac, we opted to implement a
conservative static comparability analysis using the Common Compiler
Infrastructure: Code Model and AST
API\footnote{https://cciast.codeplex.com/}. The analysis works in
three steps:

\begin{enumerate}
\item For each method, calculate which expressions (fields,
  parameters, and return values) are used together in a binary
  operation or assignment statement.
\item Until a fix point is reached, for each callsite, update the
  caller's comparability information using information from the method
  being called (the callee).
\item For each type, calculate the comparability set by merging the
  comparability sets of its methods.
\end{enumerate}

For calls to external assemblies (i.e., methods that don't have a
comparability summary), the analysis assumes that all method arguments
with compatible types are in the same comparability set.
%
Two types are considered to be compatible if either either type is
assignable to the other.

\subsubsection{Language-Specific Features and Daikon Enhancements}

To support the .NET languages, we included features in Celeriac that
aren't present in the other Daikon front-ends. Additionally, we
introduced new metadata to Daikon to support the features.

% \paragraph{Properties}

% A .NET property is a hybrid of a field and method. Each property has an
% associated \verb|get| and \verb|set| method. These methods are
% accessed without parenthesis, as if they were a field:

% \begin{verbatim}
%     myObject.Property = ...
%     var x = myObject.Property
% \end{verbatim}

% In many cases, the property is backed by a private field. Since this
% is a common case, the .NET languages provide auto-properties, which
% automatically generate the backing field. Unlike fields however,
% properties may throw exceptions, produce different values each time
% they are called (e.g., \verb|Date.Now|), or even perform
% side-effects.
% %
% To enable pretty printing and filtering of properties in Daikon, we
% introduced a \verb|is_property| variable flag.

%\paragraph{Readonly Variables}

%The .NET languages include a \verb|readonly| keyword that specifies
%that a variable must be assigned in the constructor, or given a
%constant value; the equivalent in Java is the \verb|final|
%keyword.
%
%For readonly variables, Daikon produces redundant information for
%method exit points stating that the variable has not been modified
%(e.g., for a variable \verb|x|, \verb|x == orig(x)|).
%
%We introduced an \verb|is_readonly| variable flag to Daikon to filter
%out these cases from the Daikon output.
%
%In .NET, the \verb|==| operator performs a reference-equality
%comparison for expressions with reference-type, and a value-based
%comparison for expressions with value-types
%
%\footnote{In C\#, the == and != operators can be
%  overloaded to perform a value equality check for a reference-type;
%  we do not consider this case.}.
%
%Therefore, the semantics of the \verb|is_readonly| flag depends on the
%type of the expression.

%The \verb|readonly| keyword is shallow. For reference types, the
%keyword prevents the reference from being reassigned, but does not
%prevent the object from being modified through the reference.
%
%Therefore, when considering a composite expression (e.g.,
%\verb|this.foo.bar|), Celeriac cannot naively use the \verb|readonly|
%attribute of the last field.
%%
%To compute whether an expression should be flagged as
%\verb|is_readonly|, Celeriac starts at the root (e.g., \verb|this|in
%the the case of \verb|this.foo.bar|) of the composite expression and
%propagates reference immutability and value immutability
%information. Reference immutability is propogated via the
%\verb|readonly| fields (\verb|this| is reference-immutable). Value
%immutability is propogated / introduced for \verb|readonly| fields
%that have an immutable type. We define immutability conservatively ---
%an immutable type is one that is composed of \verb|readonly| fields
%with immutable types.

% Celeriac uses a conservative immutability heuristic: reference types
% that only consist of other immutable reference types or value types
% are considered immutable.
% %
% Properties, however, complicate the matter since a one of a type's
% properties may not be idempotent even if all of the fields are
% readonly (i.e., the property can reference another non-immutable
% object); Celeriac does not attempt to handle this case.

% \paragraph{Enumerations}
% \todo{Why is this interesting?}
% %
% Enumerations in .NET are backed by integers. In fact, uses of the
% enumeration constants are replaced by integer constants in the IL (the
% comparability analysis infers which integers are enumeration contants
% based on co-occurence with a variable with an enumeration type).
% %
% To aid pretty-printing of contracts involving enumerations, we
% introduced an \verb|is_enum| variable flag to Daikon.

 \paragraph{Exceptions}
 Some specification languages such as JML allow developers to specify
 exceptional postconditions --- postconditions that are true when a
 given exception is thrown (note that this is may be different than the
 conditions that cause the exception to be thrown).
 %
 To support reasoning about exceptional postconditions, Celeriac
 produces a separate output program point for each exception type
 thrown by a method. Currently, however, this information cannot be
 fully integrated into Daikon.
 %
 Daikon has a fixed notion of the relationship between preconditions,
 postconditions, and object invariants. Each return location of a
 method has its own program point in Daikon. Invariants that appear at
 every return location are aggregate into the set of postconditions for
 the method. Additionally, preconditions and postconditions that appear
 for every method are automatically lifted to an object invariant
 program point.
 %
 To correctly support exceptions, an additional layer would need to be
 introduced to separately aggregate the regular and exceptional return
 locations of a method.
%
 For future work, we plan to use the information to detect cases in
 which a method's exceptional postconditions are not consistent with
 its normal postconditions --- e.g., the method does not clean up the
 object's internal state before exiting with an exception.

% \paragraph{F\# Lists}
% \todo{How does we support F\# lists?}

\subsection{Results}

\subsubsection{Domain-Independence}

Table~\ref{table:celeriac} shows the inferred behavioral properties
across three projects using scheme similar to that used for
Table~\ref{table:cc-usage-table}.
%
The results indicate a higher ratio of application-specific behaviors
(e.g., expression comparison, membership, universal quantification,
and implication) than as recorded by developers.
%
However, the absolute number of general behavioral properties still
represents a sizeable amount of annotation effort. For example, there
are a large number of equalities that a develop could potentially
specify as a postcondition for a method.

\begin{table*}
\begin{center}{\small %% DO NOT EDIT! THIS FILE IS AUTOMATICALLY GENERATED
\begin{tabular}{lccccccccccccccc}
\multicolumn{16}{c}{Behavior Inferred from Dynamic Traces} \\ \toprule
Contract Type & \multicolumn{3}{c}{Labs Framework} & \multicolumn{3}{c}{Mishra Reader} & \multicolumn{3}{c}{Sando} & \multicolumn{6}{c}{Quick Graph}\\
 & \multicolumn{3}{c}{Labs} & \multicolumn{3}{c}{View Models} & \multicolumn{3}{c}{Indexer} & \multicolumn{3}{c}{Algorithms} & \multicolumn{3}{c}{Collections}\\
 & {\tiny REQ} & {\tiny ENS} & {\tiny INV} & {\tiny REQ} & {\tiny ENS} & {\tiny INV} & {\tiny REQ} & {\tiny ENS} & {\tiny INV} & {\tiny REQ} & {\tiny ENS} & {\tiny INV} & {\tiny REQ} & {\tiny ENS} & {\tiny INV}\\ \midrule
Nullness & 1153 & 1265 & 235 & 2515 & 3169 & 449 & 506 & 705 & 212 & 424 & 255 & 65 & 424 & 255 & 65 \\
NullOrBlank & 102 & 387 & 44 & 70 & 593 & 26 & 194 & 492 & 2 &  &  &  &  &  &  \\
\rowcolor{Gray}
Ref. Equals & 1 & 1868 &  & 42 & 5986 &  &  & 755 &  & 1 & 423 &  & 1 & 423 &  \\
\rowcolor{Gray}
Val. Equals &  & 497 &  & 5 & 2105 &  & 1 & 849 &  &  & 123 &  &  & 123 &  \\
Zero &  &  &  &  &  & 2 & 2 & 2 & 4 &  &  &  &  &  &  \\
Non-Negative &  &  &  &  &  &  &  &  &  &  &  &  &  &  &  \\
Positive &  &  &  &  &  &  & 10 & 10 & 3 & 4 & 9 &  & 4 & 9 &  \\
Expr. Comparison &  &  &  &  & 3 &  &  & 86 &  & 1 & 13 &  & 1 & 13 &  \\
\rowcolor{Gray}
Non-Empty & 4 & 1 & 4 & 34 & 40 & 5 & 22 & 23 & 7 &  &  &  &  &  &  \\
\rowcolor{Gray}
Bounds Check &  &  &  & 1 & 1 & 7 & 25 & 23 & 16 & 5 & 2 & 6 & 5 & 2 & 6 \\
\rowcolor{Gray}
Membership &  &  &  &  &  &  & 1 & 2 &  &  &  &  &  &  &  \\
\rowcolor{Gray}
ForAll & 9 & 49 &  & 437 & 552 & 54 & 227 & 532 & 293 & 1 & 1 &  & 1 & 1 &  \\
Indicator & 174 & 588 & 39 & 568 & 3031 & 142 & 112 & 481 & 27 &  &  &  &  &  &  \\
Implication &  & 101 &  &  & 379 &  &  & 18 &  &  & 92 &  &  & 92 &  \\
Frame Condition &  & 54 &  &  & 761 &  &  & 434 &  &  & 1 &  &  & 1 &  \\
\rowcolor{Gray}
Other & 110 & 160 & 27 & 379 & 655 & 49 & 156 & 254 & 25 & 8 & 25 &  & 8 & 25 &  \\
\bottomrule
\end{tabular}
}
  \caption{Behavioral properties dynamically inferred by Daikon; the
    contents of each category are mutually exclusive.  The Mishra
    Reader View Models component and Sando Indexer component are the subjects of the developer case study in
    Section~\ref{sec:observe}.
    %
    \todo{The NullOrBlank, Non-Empty, Bounds Check, and Indicator
      categories need to be fixed.}}
\end{center}
\label{table:celeriac}
\end{table*}

\subsubsection{Expressivity}
To get an intuition for how often different methods exhibit the same
behavior, we calculated the frequency at which
properties were found to hold for multiple methods within a class
(Figure~\ref{abbrev-histogram}). Properties appearing for every method
in a class are not shown, as they would be inferred to be an object
invariant.
%
This data does not directly validate the benefit of abbreviator
methods, however. The value of an abbreviator is related to both the
number of contracts specified in the grouping, and the number of
methods for which the grouping applies.

Finding the optimal set of method abbreviators is not straight-forward
since an abbreviator method can refer to another abbreviator method.
To approximate the expected benefit, calculated the savings (in terms
of number of contracts and abbreviator calls written) when greedily
introducing abbreviators. \todo{Results}

%\begin{figure}
%    \includegraphics[width=0.5\textwidth]{study/celeriac/data/abbrev-histogram}
%\caption{A histogram showing the frequency at which instance-expression properties are
%  inferred for one or more methods in a class (for AutoDiff).
%  Properties appearing for every method are not included, as they are
%  reported as object invariants.
%  %
%  If two or more properties co-occur for multiple methods, the
%  developer can group the properties using a Contract Abbreviator
%  method.}
%\label{fig:abbrev-histogram}
%\end{figure}
%\begin{figure}
%  \includegraphics[width=0.5\textwidth]{study/celeriac/data/impact-histogram}
%  \caption{A histogram showing the ratio of data behavior properties
%    that can be eliminated for a project by greedily creating Contract
%    Abbreviator methods (for AutoDiff). The data indicates that for
%    most classes abbreviator methods provide no benefit, but for some
%    classes abbreviator methods can greatly reduce the of contracts
%    needed to express inferred behaviors.}
%\label{fig:impact-histogram}
%\end{figure}

\subsubsection{Difference Between Contracts and Observed Behavior}



%\subsubsection{Threats to Validity}
%\label{sec:celeriac-threats}
%
%\paragraph{Contract Precision}
%
%In \cite{NimmerE02:ISSTA}, Nimmer and Ernst reported that contracts
%inferred from even small test suites were relatively precise, with
%less than 10\% on average being incorrect (from a verification
%perspective). Polikarpova et al. later reported in
%~\cite{Polikarpova2009} that one third of contracts inferred by Daikon
%were incorrect or irrelevant for larger software written in Eiffel.
%
%The Eiffel trace-generator, however, does not perform comparability
%analysis.

%Daikon's object-oriented support is limited to object invariants -- it
%does not incorporate inheritance information, and therefore does not
%consider the implication of contracts on behavioral subtyping. As,
%Csallner and Smaragdakis point out inheritance can cause Daikon to
%output internally inconsistent contracts~\cite{Csallner06}.

\paragraph{Contract Recall}

Since Daikon works by instantiating contract templates and
invalidating the templates against the trace, it's output is directly
tied to the ``grammar'' of contracts provided.

Additionally, since Daikon only calls methods that are known to be
pure, missing pure methods would result in contracts involving those
methods being missed. Celeriac's purity assumptions are conservative,
only including auto-generated property getters and methods annotated
with the \verb|Pure| annotatation (only libraries annotated with Code
Contracts include the \verb|Pure| annotation.


\section{Conclusion}

\newpage
\bibliographystyle{plain}
\bibliography{thesis}

\end{document} 