\documentclass{article}

\usepackage{code}
\usepackage{url}
\usepackage{geometry}
\usepackage{graphicx}
\usepackage{caption}
\usepackage{verbatim}
\usepackage{fancyvrb}
\usepackage{color}

\newcommand\ch[1]{\textcolor[rgb]{0,0,1}{\textbf{#1}}}
\newcommand\clh[1]{\textcolor[rgb]{0,.5,1}{\textbf{#1}}}
\newcommand\coh[1]{\textcolor[rgb]{0,.6,0}{\textbf{#1}}}

\newcommand{\timetbl}[4]{\par\vspace{5mm}\begin{tabular}{ l l l l }
Estimated Completion Date: & #1 & Estimated Completion Time: & #2 \\
Actual Completion Date: & #3 & Actual Completion Time: & #4 \\
\end{tabular}}

\title{Celeriac: A Daikon .NET Front End}
\author{Kellen Donohue}

\begin{document}
\maketitle

\section{Overview}
\subsection{Contracts}
Contracts provide an agreement between a method implementer and caller, and also describe well-formed object. The following code snippet shows how a typical bank account withdraw method may be annotated with contracts.
\begin{center}
\begin{Verbatim}[commandchars=\\\{\}]

\ch{class} BankAccount \{
  \ch{bool} Withdraw(\clh{Account} acct, \ch{decimal} amt) \{
      \coh{// Don't withdraw a negative amount}
      \clh{Contract}.Requires(amt >= 0);
      \coh{// Return whether withdraw succeeded}
      \clh{Contract}.Ensures(
        \clh{Contract}.Result<\ch{bool}>() == acct.bal >= amt);
      \coh{// Update the account properly}
      \clh{Contract}.Ensures(acct.bal == \clh{Contract}.OldValue(acct.bal) - \clh{Contract}.Result<\ch{bool}>() ? amt : 0);
      ...
  \}
\}
\end{Verbatim}
\end{center} \captionof{figure}{Bank Account}
\~ \\
Contracts are useful for ensuring the correct behavior of program.
\section{Related Work}
\section{User Documentation}
\section{Developer Documentation}
\section{Research Experiments}
\section{Conclusion}

\newpage
\bibliographystyle{plain}
\bibliography{thesis}

\end{document}