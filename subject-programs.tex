\subsection{Subject Programs}
\label{sec:subject-programs}

In this section, we describe the four subject programs and their
current use of the Code Contract tool as described by a developer
questionnaire. We selected these subject projects from the ``Projects
Using Code Contracts'' listing on the Microsoft Code Contracts
project
website\footnote{https://research.microsoft.com/en-us/projects/contracts/}.
%
The projects were chosen because they are open-source, actively
developed, have users (i.e., are not toy programs), and use contracts
in a meaningful way.

\subsubsection{Labs Framework}

The Labs Framework\footnote{https://labs.codeplex.com/} (11K SLOC) is
a framework for managing ``experiments'' exploring the behavior of an
API are library. Labs provide a means for demonstrating features to a
library's users, or for a user to perform reproducible
experimentation.
\\ \\
The project uses \verb|ccheck|, the static Code Contract checker,
which is enabled by default for both the Debug and Release
configurations (but not the Profiling configuration).

\subsubsection{Mishra Reader}

Mishra Reader\footnote{\url{http://mishrareader.codeplex.com/}}
(19K SLOC) is an open-source Google Reader client with over 27,500
downloads.
%
The developer we interviewed introduced Code Contracts to the core
library 2 1/2 years ago (Fall 2011) to help reduce bugs in
multi-threaded code.
% Before this study, the codebase included 59 preconditions, 14 postconditions, and 4
% object invariants. 69 of these contracts were nullness checks on
% either an argument or method return value; the project did not include quantification
% (\verb|ForAll| or \verb|Exists|) contracts.
%
Code Contracts were written after the methods were written to aide in
debugging (i.e., as opposed to design by contract).
%
Initially satisfied with Code Contracts, the developer became
dissatisfied due to a lack of support for debugging with contract in
\verb|async| and \verb|await| constructs (the IL rewriter did not
properly modify the debugging information); debugging support for
these constructs has since been added.

The development team does not use the static Code Contract checker,
citing that it is slow and noisy. Runtime checks are configured to
throw an exception and terminate the program; the project's test suite
is also set up to fail a test given a contract violation.

\begin{table}[t]
\begin{center}{
\begin{tabular}{ ll }
  Project & Code Contract Use \\ \hline
  Labs Framework & Static checking \\
  Mishra Reader & Debugging concurrent code \\
  Sando & Early runtime error detection \\
  QuickGraph & Static checking and Pex
\end{tabular} }
\end{center}
\caption{The subject programs and their primary use of Code Contracts}
\end{table}

\subsubsection{Sando}

Sando\footnote{https://sando.codeplex.com/} is a Lucene-based code
search engine that includes a Visual Studio interface; the software is
Beta with over 500 downloads.
%
The Sando project has used Code Contracts for 1 1/2 years (since
Winter 2012). Code Contracts were introduced to the project since one
of main contributors had seen a webinar on Code Contracts and wanted
to try them out. The team found the initial project setup more
difficult than expected, especially due to issue integrating Code
Contracts into the automated TeamCity build. Despite these initial
challenges, the developer we interviewed feels that Code Contracts has
sped the discovery of bugs or regressions, as well as increasing
confidence in the quality of code containing contracts. Code Contracts
are enabled only for debug builds.
% The Sando project does not include any XML documentation, instead
% using method and parameter naming as the primary documentation (along
% with some comments within methods).

The team primarily uses Code Contracts in core functionality,
primarily in the Index component, as placing bad data into the index
can result in later errors. The interview developer typically writes
contracts after making a change, but before running the unit test
suite before check-in.

The interviewed developer was not aware of the static analyis tools
Code Contracts. The project does not use any static analysis tools, in
part because the team has limited build engineering resources. Code
Contracts is seen as offering additional quality assurance without
requiring additional build engineering, and likely makes the team less
likely to try other quality assurance tools.

\subsubsection{Quick Graph}

Quick Graph\footnote{https://quickgraph.codeplex.com/} (32K SLOC) is
a data structure and algorithm library. The project uses both the Code
Contract checker, and PEX.
